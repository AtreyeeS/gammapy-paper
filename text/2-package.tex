\section{Gammapy package}
\label{sec:package}

The \gammapy package is structured into multiple sub-packages.

\subsection{gammapy.data}
DL3 level data

\subsection{gammapy.makers}
Data reductionn

\subsection{gammapy.datasets}
DL4 level data

\subsection{gammapy.modelling}
Models and fitting

\subsection{gammapy.estimators}
Estimators

\subsection{gammapy.visualisation}
Plotters etc.
\subsection{gammapy.analysis}
High level analysis API

\subsection{gammapy.astro}
Dark matter models, source population modelling


\subsection{gammapy.catalog}
\subsection{gammapy.maps}
\subsection{gammapy.irf}
\subsection{gammapy.utils}



Outline:
* List typical analysis use cases
* Can use from Python and Jupyter -> show Figure with Jupyter notebook here.
* Gammapy code structure
* How Numpy and Astropy is used


Figures:
* Add a Figure showing dataflow in a typical application
DL3 at the top, spectrum, map, lightcurve, fit results at the bottom.
Mention major classes in between (DataStore, EventList, Map, MapMaker, MapFit, …)
* Probably not: Figure showing sub-packages and how they relate (gammapy.data and gammapy.irf at the base, then gammapy.maps, etc.
* The code example Figure how to make a counts map, to explain how the package works.
